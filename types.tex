\documentclass{beamer}

\usetheme{default}

\title{Introduction to Type Inference}
\author{Brian Hurt}
\date{22 April 2015}

\begin{document}

\begin{frame}
    \titlepage
\end{frame}

\begin{frame}
\begin{center}
\Large{
Source for this presentation (latex + beamer):

}
\end{frame}

\begin{frame}
\frametitle{Why you care}
\begin{itemize}
\item Forms the basis for most static type systems (Haskell, Ocaml, Scala, Swift, etc.)
\item Increasingly used in optional type systems (Typed Racket, Clojure's
core.typed, C++'s auto, static analysis tools like Coverity, etc.)
\item The basis for more advanced type system features
\item Helps in understanding otherwise cryptic error messages
\end{itemize}

\end{frame}

\begin{frame}
\frametitle{What this talk does}
\begin{itemize}
\item Introduces a simple subset of Haskell called SiML
\item Creates a type inference/type checking algorithm for it
\item Concurrently introduces the notation used by the research (the
``Natural Deduction Style'')
\end{itemize}

This code is not how GHC really works, however...
\end{frame}

\begin{frame}
\frametitle{Technobabble}
\begin{center}
SiML is an enriched lambda caculus on which we perform a modified and
simplified Hindley-Milner-Damas algorithm.
\end{center}
\end{frame}

\begin{frame}[fragile]
\frametitle{The SiML Language}
{\tt \Large{
\begin{verbatim}
    data Expr a =
        Const Const
        | Var Var
        | If (Expr a) (Expr a) (Expr a)
        | Apply (Expr a) (Expr a)
        | Lambda Var (Expr a)
        | Let Var (Expr a) (Expr a)
        | LetRec [ (Var, (Expr a)) ]
                (Expr a)
        | Typed (Expr a) (Type a)
        deriving (Read, Show, Eq, Ord)
\end{verbatim}
}}
\end{frame}


\begin{frame}[fragile]
\frametitle{The SiML Language}
{\tt \Large{
\begin{verbatim}
    instance Functor Expr where ...

    instance Foldable Expr where ...
\end{verbatim}
}}
\end{frame}

\begin{frame}[fragile]
\frametitle{The SiML Language}
{\tt \Large{
\begin{verbatim}
    type Var = String

    data Const =
        ConstInt Integer
        | ConstBool Bool
        deriving (Read, Show, Eq, Ord)
\end{verbatim}
}}
\end{frame}

\begin{frame}[fragile]
\frametitle{The SiML Language}
{\tt \Large{
\begin{verbatim}
    data Type a =
        TBool
        | TInt
        | TFun Type Type
        | TVar a
        deriving (Read, Show, Eq, Ord)

    instance Functor Type where ...

    instance Foldable Type where ...

\end{verbatim}
}}
\end{frame}

\begin{frame}[fragile]
\frametitle{The SiML Language}
{\tt\Large{
\begin{verbatim}
    data Stmnt =
        LetStmnt Var Expr
        | LetRecStmnt [ (Var, Expr) ]
\end{verbatim}
}}
\end{frame}

\begin{frame}
\frametitle{Basic Intuition}
\begin{center}
\huge{
The basic intuition:

\vspace{20pt}
The structure of the code itself imposes constraints on the types.

\vspace{20pt}
We can use these constraints to determine the type.
}
\end{center}
\end{frame}

\begin{frame}
\frametitle{Basic Intuition}
\begin{center}
\huge{
For example, given:

$$\text{if}\hspace{5pt}e_1\hspace{5pt}\text{then}\hspace{5pt}e_2
\hspace{5pt}\text{else}\hspace{5pt}e_3$$

Where:
$$e_1:t_1\qquad{}e_2:t_2\qquad{}e_3:t_3$$}
\end{center}
\end{frame}

\begin{frame}
\frametitle{Basic Intuition}
\begin{center}
\huge{
Then we know that:

$$t_1 = Bool$$

$$t_2 = t_3 = \text{type of the if statement}$$}
\end{center}
\end{frame}


\begin{frame}
\frametitle{Natural Deduction Style}
\begin{center}
\huge{
The ``natural deduction style'' of logical systems.
}
\end{center}
\end{frame}

\begin{frame}
\frametitle{Natural Deduction Style}
\Huge{
$$\frac{\Gamma,x:t_1\dashv{}e:t_2}{\Gamma\dashv{}\lambda{}x.e:t_1\rightarrow{}t_2}$$
}
\end{frame}

\begin{frame}
\frametitle{Natural Deduction Style}
\begin{center}
\Huge{
Don't Panic
}
\end{center}
\end{frame}

\begin{frame}
\frametitle{Natural Deduction Style}
\huge{
$$\frac{\text{If this is true}}{\text{Then this is true}}$$
}
\end{frame}

\begin{frame}
\frametitle{Natural Deduction Style}
\Large{
$$\frac{e_1:t_1\quad{}e_2:t_2\quad{}e_3:t_3\quad{}t_1 = \text{Bool}\quad{}t_2 = t\quad{}t_3 = t}{
(\text{if}\hspace{5pt}e_1\hspace{5pt}\text{then}\hspace{5pt}e_2
\hspace{5pt}\text{else}\hspace{5pt}e_3):t}$$
}
\end{frame}

\begin{frame}
\frametitle{Type Unification}
\begin{center}
\Huge{
What is meant by:

$$t_1 = t_2$$
}
\end{center}
\end{frame}

\begin{frame}
\frametitle{Type Unification}
\begin{center}
\Huge{
The mathematician means:

\vspace{40pt}

You can replace all occurrences of $t_1$ with $t_2$.
}
\end{center}
\end{frame}

\begin{frame}
\frametitle{Type Unification}
\Large{
So you can replace $t_1$ with Bool, and $t_2$ and $t_3$ with $t$:

$$\frac{e_1:\text{Bool}\qquad{}e_2:t\qquad{}e_3:t}{
(\text{if}\hspace{5pt}e_1\hspace{5pt}\text{then}\hspace{5pt}e_2
\hspace{5pt}\text{else}\hspace{5pt}e_3):t}$$
}
\end{frame}

\begin{frame}
\frametitle{Type Unification}
\begin{center}
\Huge{
The programmer means:

\vspace{40pt}

Change the state of the system so that $t_1 = t_2$.
}
\end{center}
\end{frame}

\begin{frame}[fragile]
\frametitle{Type Unification}

{\tt\large{
\begin{verbatim}
    type Matching a = ...

    instance Monad Matching where
        ...

    unify :: Type Var -> Type Var
             -> Matching (Type Var)
    unify = undefined
\end{verbatim}

So $t_1 = t_2$ becomes {\tt unify t1 t2}.
}}
\end{frame}

\begin{frame}[fragile]
\frametitle{Type Inference: If }

{\tt\large{
\begin{verbatim}
    typeInfer :: Expr Var
                    -> Matching (Type Var)

    typeInfer (If e1 e2 e3) = do
        t1 <- typeInfer e1
        t2 <- typeInfer e2
        t3 <- typeInfer e3
        _ <- unify t1 TBool
        unify t2 t3
\end{verbatim}
}}
\end{frame}

\begin{frame}
\frametitle{Type Inference: Constants}
\Large{
Integer and boolean constants have their obvious types.

$$\frac{}{\text{True}:\text{Bool}\qquad{}\text{False}:\text{Bool}}$$

\vspace{40pt}

$$\frac{}{0:\text{Int}\quad{}1:\text{Int}\quad{}...}$$
}
\end{frame}

\begin{frame}[fragile]
\frametitle{Type Inference: Constants}

{\tt\large{
\begin{verbatim}
    typeInfer (Const (ConstInt _)) =
        return TInt
    typeInfer (Const (ConstBool _)) =
        return TBool
\end{verbatim}
}}
\end{frame}

\begin{frame}
\frametitle{Type Inference: Constants}
\Large{
Type application is also obvious:

$$\frac{f:t_1 \rightarrow{} t_2\qquad{}x:t_1}{(f\hspace{5pt}x):t_2}$$
}
\end{frame}

\begin{frame}[fragile]
\frametitle{Type Inference: Application}

\Large{
{\tt
\begin{verbatim}
    typeInfer (Apply f x) = do
        tf <- typeInfer f
        tx <- typeInfer x
        case tf of
            TFun t1 t2 -> do
                _ <- unify t1 tx
                return t2
            _ -> fail "Not a function!"
\end{verbatim}
}}
\end{frame}

\begin{frame}
\frametitle{Type Inference: Typed Expressions}
\Large{

Typed expressions are also obvious:

$$\frac{x:t}{(x\hspace{5pt}::\hspace{5pt}t):t}$$
}
\end{frame}

\begin{frame}[fragile]
\frametitle{Type Inference: Typed Expressions}

\Large{
{\tt
\begin{verbatim}
    typeInfer (Typed x t) = do
        tx <- typeInfer x
        unify tx t
\end{verbatim}
}}
\end{frame}

\begin{frame}[fragile]
\frametitle{Type Inference: Var and Let}

Consider...

{\tt\large{
\begin{verbatim}
    typeInfer (Var x) = ...
    typeInfer (Let x e1 e2) = ...
\end{verbatim}
}}
\end{frame}

\begin{frame}
\frametitle{Type Inference: Var and Let}
\begin{center}
\Large{
We need to pass around a map of variables to their types.
}
\end{center}
\end{frame}

\begin{frame}[fragile]
\frametitle{Type Inference: Var and Let}

{\tt\large{
\begin{verbatim}
    typeInfer :: [ (Var, Type Var) ]
                    -> Expr Var
                    -> Matching (Type Var)

    typeInfer ctx (Var x) = 
        case (lookup x ctx) of
            Just t -> return t
            Nothing -> fail "Unknown variable"

    typeInfer ctx (Let x e1 e2) = do
        t1 <- typeInfer ctx e1
        typeInfer ((x, t1) : ctx) e2
\end{verbatim}
}}
\end{frame}


\begin{frame}[fragile]
\frametitle{Type Inference: If and Constants (v2.0)}

{\tt\large{
\begin{verbatim}
    typeInfer ctx (If e1 e2 e3) = do
        t1 <- typeInfer ctx e1
        t2 <- typeInfer ctx e2
        t3 <- typeInfer ctx e3
        _ <- unify t1 TBool
        unify t2 t3
    typeInfer _ (Const (ConstInt _)) =
        return TInt
    typeInfer _ (Const (ConstBool _)) =
        return TBool
\end{verbatim}
}}
\end{frame}

\begin{frame}
\frametitle{Natural Deduction Style: Contexts}
We use $\Gamma$ to represent our context, and $\dashv$ to mean ``evaluate
the right hand side with the context on the left'':

\LARGE{
$$\frac{\Gamma\dashv{}e_1:Bool\qquad\Gamma\dashv{}e_2:t\qquad\Gamma\dashv{}e_3:t}{\Gamma\dashv{}(\text{if}\hspace{5pt}e_1\hspace{5pt}\text{then}\hspace{5pt}e_2
\hspace{5pt}\text{else}\hspace{5pt}e_3):t}$$
}
\end{frame}

\begin{frame}
\frametitle{Natural Deduction Style: Contexts}
\LARGE{
$$\frac{x:t \in{} \Gamma}{\Gamma{}\dashv{}x:t}$$

\vspace{20pt}

$$\frac{\Gamma\dashv{}e_1:t_1\qquad{}\Gamma{},x:t_1\dashv{}e_2:t_2}{\Gamma\dashv{}(\text{let}\hspace{5pt}x=e_1\hspace{5pt}\text{in}\hspace{5pt}e2):t_2}$$
}
\end{frame}

\begin{frame}
\frametitle{Type Inference: Lambda}
\begin{center}
\Large{
This works great for let (where we know the value being bound, and therefor
the type).

\vspace{20pt}

But what about lambda?}
\end{center}
\end{frame}

\begin{frame}[fragile]
\frametitle{Type Inference: Lambda}
\Large{
We need to know the type of the argument before we can infer the
type of the body- except it's how the argument is used which
determines it's type!  For example:

\tt
\begin{verbatim}
    (\x -> x + 1)
\end{verbatim}
}
\end{frame}

\begin{frame}
\frametitle{Type Inference: Lambda}

\Large{
With math, we can just assume it exists:

$$\frac{\Gamma,x:t_1\dashv{}e:t_2}{\Gamma\dashv{}(\lambda{}x.e):t_1 \rightarrow{} t_2}$$

\vspace{20pt}

With code, we can't pull this stunt.
}
\end{frame}

\begin{frame}[fragile]
\frametitle{Type Inference: Lambda}

{\tt\Large{
\begin{verbatim}
    typeInfer ctx (Lambda x e) = do
        t1 <- what goes here?
        t2 <- typeInfer ((x, t1) : ctx) e
        return (TFun t1 t2)
\end{verbatim}
}}
\end{frame}

\begin{frame}
\frametitle{Type Inference: Lambda}
By the way, does this expression look familiar?

{\Huge
$$\frac{\Gamma,x:t_1\dashv{}e:t_2}{\Gamma\dashv{}\lambda{}x.e:t_1\rightarrow{}t_2}$$
}

\vspace{40pt}

(hint: Don't Panic)
\end{frame}

\begin{frame}
\frametitle{Type Variables}
\begin{center}
\Huge{
The Two Types of

Type Variables
}
\end{center}
\end{frame}

\begin{frame}
\frametitle{Type Variables: Universal}
\Large{
\begin{definition}
A \alert{universal} type variable can be any type.

Also known as: rigid type variables, skolem type variables.
\end{definition}
}
\end{frame}

\begin{frame}[fragile]
\frametitle{Type Variables: Universal}
\Large{
\begin{center}
Universal type variables are the ``normal'' type variables:
\end{center}

{\tt
\begin{verbatim}
    map :: (a -> b) -> [a] -> [b]
\end{verbatim}
}

}
\end{frame}

\begin{frame}
\frametitle{Type Variables: Existential}
\Large{
\begin{definition}
A \alert{existential} type variable represents a specific type that is
not yet known.  The type is known when the type variable is unified with
some other type.

Also known as: flexible type variables.

\end{definition}
}
\end{frame}

\begin{frame}[fragile]
\frametitle{Type Variables: Universal}
\large{
\begin{center}
You don't see existential type variables in Haskell, but other languages
do display them:
\end{center}

{\tt
\begin{verbatim}
> ocaml
        OCaml version 4.01.0

# let r = ref None;;
val r : '_a option ref = {contents = None}
# r := Some 1;;
- : unit = ()
# r;;
- : int option ref = {contents = Some 1}
# 
\end{verbatim}
}

}
\end{frame}

\begin{frame}
\frametitle{Type Variables}
\Large{
\alert{Universal} type variable: this type could be any type in the
whole wide \alert{universe}.

\vspace{40pt}

\alert{Existential} type variable: this type \alert{exist}s, but we
don't know what it is yet.
}
\end{frame}

\begin{frame}[fragile]
\frametitle{Type Inference: Lambda}

\Large{
\begin{center}
Existential type variables solve our Lambda problem.
\end{center}

{\tt
\begin{verbatim}
    typeInfer ctx (Lambda x e) = do
        t1 <- allocExistVar
        t2 <- typeInfer ((x, t1) : ctx) e
        return (TFun t1 t2)
\end{verbatim}
}}
\end{frame}

\begin{frame}[fragile]
\frametitle{Type Inference: Lambda}

Of course, this requires some type signature changes:

{\tt
\begin{verbatim}
    type EVar = ...

    type TVar = Either Var EVar

    allocExistVar :: Matching (Type TVar)
    allocExistVar = undefined

    unify :: Type TVar -> Type TVar
             -> Matching (Type TVar)
    unify = undefined

    typeInfer :: [ (Var, Type TVar) ]
                    -> Expr Var
                    -> Matching TVar
    ...
\end{verbatim}
}
\end{frame}

\begin{frame}[fragile]
\frametitle{Type Inference: Lambda}

Existential type variables also solves the problem with let rec:

{\tt
\begin{verbatim}
    typeInfer ctx (LetRec defns e) = do
            ts <- mapM getEVar defns
            let ctx' = ts ++ ctx
            mapM_ (inferBody ctx') defns
            typeInfer ctx' e
        where
            getEVar (v, _) = do
                t <- allocExistVar
                return (v, t)
            inferBody c (_, e1) =
                typeInfer c e1
\end{verbatim}
}
\end{frame}

\begin{frame}
\frametitle{Type Variables: Two Problems}
\Huge{
\begin{center}
Two Problems
\end{center}
}
\end{frame}

\begin{frame}
\frametitle{Type Variables: Problem 1: Typed Expressions}
\Large{
\begin{center}
With Typed expressions, the AST gives us {\tt Type Var}, but we
need {\tt Type TVar} to pass in to unify.

\vspace{40pt}

Note: the types given in the AST can only contain universal type
variables!
\end{center}
}
\end{frame}

\begin{frame}[fragile]
\frametitle{Type Variables: Problem 1: Typed Expressions}

\Large{
Using {\tt fmap Left} converts a {\tt Type Var} into a
{\tt Type TVar} (making all variables universal):

{\tt
\begin{verbatim}
    typeInfer (Typed x t) = do
        tx <- typeInfer x
        unify tx (fmap Left t)
\end{verbatim}
}}
\end{frame}

\begin{frame}[fragile]
\frametitle{Type Variables: Problem 2: Using Variables}
\Large{
Consider map:

{\tt
\begin{verbatim}
    map :: (a -> b) -> [a] -> [b]
\end{verbatim}
}

\vspace{20pt}

\alert{Inside} map, {\tt a} and {\tt b} are universal type variables, and
can not be unified with any other type.

\vspace{20pt}

But when we \alert{call} map, they can be any type we want.
}
\end{frame}

\begin{frame}[fragile]
\frametitle{Type Variables: Problem 2: Using Variables}
\Large{
\begin{center}
When we use a variable whose type has universal type variables, the
universal type variables need to be converted into existential type
variables.

\vspace{20pt}

But, all instances of the same universal type variable need to map
to the same existential type variable.
\end{center}
}
\end{frame}

\begin{frame}[fragile]
\frametitle{Type Variables: Problem 2: Using Variables}

{\tt
\begin{verbatim}
    import Data.List(nub)
    import Data.Foldable(toList)
    import Data.Maybe(fromJust)

    typeInfer ctx (Var x) =
        case (lookup x ctx) of
            Just t -> do
                let uvars = nub (toList t)
                evars <- mapM (const allocExistVar)
                                            uvars
                let varMap = zip uvars evars
                return (fmap (fixVar varMap) t)
            Nothing -> fail "Unknown variable"
        where
            fixVar varMap v =
                fromJust (lookup v varMap)
\end{verbatim}
}
\end{frame}

\begin{frame}
\frametitle{Unify}
\Huge{
\begin{center}
Unify
\end{center}
}
\end{frame}


\begin{frame}[fragile]
\frametitle{Unify}

\Large{\tt
\begin{verbatim}
    data Type a =
        TBool
        | TInt
        | TFun Type Type
        | TVar a

    unify :: Type TVar -> Type TVar
             -> Matching (Type TVar)
    unify = undefined
\end{verbatim}
}
\end{frame}

\begin{frame}[fragile]
\frametitle{Unify}

\Large{

The easy cases:

{\tt
\begin{verbatim}
    unify TBool TBool = return TBool
    unify TInt  TInt  = return TInt
\end{verbatim}
}}
\end{frame}

\begin{frame}[fragile]
\frametitle{Unify}
\large{
\begin{center}
$$t_1 \rightarrow{} t_2 = t_3 \rightarrow{} t_4$$
implies:
$$t_1 = t_3\qquad{}\&\&\qquad{}t_2 = t_4$$
\end{center}

{\tt
\begin{verbatim}
    unify (TFun t1 t2) (TFun t3 t4) = do
        t5 <- unify t1 t3
        t6 <- unify t2 t4
        return (TFun t5 t6)
\end{verbatim}
}}
\end{frame}


\begin{frame}[fragile]
\frametitle{Unify}
\large{
\begin{center}
Two universal type variables only unify if they're the same type
variable.
\end{center}

{\tt
\begin{verbatim}
    unify (TVar (Left a)) (TVar (Left b))
        | a == b = return (TVar (Left a))
        | otherwise = fail "Type error"
\end{verbatim}
}}
\end{frame}

\begin{frame}
\frametitle{Unify}
\Large{
\begin{center}
Interesting Question:

\vspace{20pt}

What is the scope of a universal type variable?  That is: when does one
{\tt a} in one type expression match represent the same (polymorphic)
type as another {\tt a} in some other type expression?
\end{center}
}
\end{frame}


\begin{frame}
\frametitle{Unify: Existential Types}
\Huge{
\begin{center}
Existential Types
\end{center}
}
\end{frame}

\begin{frame}
\frametitle{Unify: Existential Types}
The rules for unifying existential types are:
\begin{itemize}
\item Existential type variables can be assigned another type \alert{at
most once}.
\item If an existential type variablehas been assigned another type
previously, we unify with that type instead.
\item Otherwise, we assign the other type to the existential type variable.
\item It is possible for both types to be existential type variables
which have not been assigned previously, in which case we assign one
to the other.
\end{itemize}
\end{frame}

\begin{frame}[fragile]
\frametitle{Unify: Existential Types}
\large{
We need some way to set an existential type variable to a given type:
{\tt
\begin{verbatim}
    setEVar :: EVar -> Type TVar -> Matching ()
    setEVar = undefined
\end{verbatim}
}

And we need a way to get the value it was set to (if it was set
previously):

{\tt
\begin{verbatim}
    getEVar :: EVar -> Matching (Maybe (Type TVar))
    getEVar = undefined
\end{verbatim}
}
}
\end{frame}

\begin{frame}[fragile]
\frametitle{Unify: Existential Types}
\large{
{\tt
\begin{verbatim}
    unify (TVar (Right a)) t2 = do
        mt1 <- getEVar a
        case mt1 of
            Some t1 -> unify t1 t2
            None -> do
                setEVar a t2
                return t2
\end{verbatim}
}}
\end{frame}

\begin{frame}[fragile]
\frametitle{Unify: Existential Types}
\large{
{\tt
\begin{verbatim}
    unify t1 (TVar (Right b)) = do
        mt2 <- getEVar b
        case mt2 of
            Some t2 -> unify t1 t2
            None -> do
                setEVar b t1
                return t1
\end{verbatim}
}}
\end{frame}

\begin{frame}[fragile]
\frametitle{Unify: Existential Types}
\large{
\begin{center}
\alert{All} other patterns are type errors!
\end{center}

{\tt
\begin{verbatim}
    unify _ _ = fail "Type error"
\end{verbatim}
}}
\end{frame}

\begin{frame}[fragile]
\frametitle{Unify: Matching Utils}
\large{
{\tt
\begin{verbatim}
    import Data.Default

    type EVar = Int

    data MState = MState {
        evarCounter :: Int,
        evarMappings :: [ (EVar, Type TVar) ]
    }

    type Matching a = State MState a

    instance Default MState where
        def = MState 0 []
\end{verbatim}
}}
\end{frame}

\begin{frame}[fragile]
\frametitle{Unify: Matching Utils}
\large{
{\tt
\begin{verbatim}
    allocExistVar :: Matching (Type TVar)
    allocExistVar = do
        mstate <- get
        let evar = evarCounter mstate
        put (mstate { evarCounter = evar + 1 })
        return (Type (Right evar))
\end{verbatim}
}}
\end{frame}

\begin{frame}[fragile]
\frametitle{Unify: Matching Utils}
\large{
{\tt
\begin{verbatim}
    setEVar :: EVar -> Type TVar -> Matching ()
    setEVar evar typ = do
        mstate <- get
        let mappings =
            (evar, typ) : evarMappings mstate
        put (mstate { evarMappings = mappings })
\end{verbatim}
}}
\end{frame}

\begin{frame}[fragile]
\frametitle{Unify: Matching Utils}
\large{
{\tt
\begin{verbatim}
    getEVar :: EVar -> Matching (Maybe (Type TVar))
    getEVar evar = do
        mstate <- get
        let mappings = evarMappings mstate
        return (lookup evar mappings)
\end{verbatim}
}}
\end{frame}

\begin{frame}
\frametitle{One Last Problem}
\begin{center}
{\Huge
How do you prevent existential type variables from ``leaking'' into
a global type?}
\end{center}
\end{frame}

\begin{frame}[fragile]
\frametitle{One Last Problem}
Before a type can be promoted to the global scope:
\begin{itemize}
\item If an existential type variable has been assigned another type,
replace the existential type variable with the assigned type.
\item If an existential type variable has not been assigned another type,
generate a new, unique universal type variable and assign it to the
existential type.
\end{itemize}

Repeat the above until the type no longer has any existential type variables
in it.
\end{frame}

\begin{frame}
\frametitle{Type Variables: Universal}
\Large{
\begin{definition}
The act of replacing an unassigned existential type variable with a new,
unique universal type variable is called \alert{Skolemization} (named after
Thoralf Skolem).
\end{definition}
}
\end{frame}

\begin{frame}[fragile]
\frametitle{Summary}
\Large{
We now have  function:

{\tt
\begin{verbatim}
    typeInfer :: [ (Var, Type Var) ]
                    -> Expr Var
                    -> Matching (Type Var)
\end{verbatim}
}

Which can be used for both type inference and type checking.

\vspace{10pt}

Comming soon: working code in my github repo.
}
\end{frame}

\begin{frame}
\frametitle{Summary}
In addition, formula like the following aren't so scary any more:

{\Huge
$$\frac{\Gamma,x:t_1\dashv{}e:t_2}{\Gamma\dashv{}\lambda{}x.e:t_1\rightarrow{}t_2}$$
}
\end{frame}

\begin{frame}
\frametitle{Summary}
\begin{center}
Where to go from here:

\vspace{20pt}

``Types and Programming Languages''
Benjamin C. Pierce
 
\vspace{20pt}

Then start reading papers.
\end{frame}

\begin{frame}
\begin{center}
\Huge{
fini
}
\end{frame}

\end{document}

