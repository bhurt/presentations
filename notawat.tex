\documentclass{beamer}

\usetheme{default}

\title{The Not-A-Wat in Haskell}
\author{Brian Hurt}
\date{18 Oct 2015}

\begin{document}

\begin{frame}
    \titlepage
\end{frame}

\begin{frame}
\begin{center}{\tt \Huge <RANT>}\end{center}
\end{frame}

\begin{frame}
\frametitle{Wat Definition}
\begin{center}\Large{What is a "Wat"?}\end{center}
\end{frame}

\begin{frame}
\frametitle{Wat Definition}
\begin{center}\Large{https://www.destroyallsoftware.com/talks/wat}\end{center}
\end{frame}

\begin{frame}
\frametitle{Wat Definition}
\begin{center}
My definition:

A "wat" (w.r.t. programming languages) is a behavior that is both
unexpected, and inconsistent with all the other behaviors of the
language.
\end{center}
\end{frame}

\begin{frame}[fragile]
\frametitle{The Wat In Question}
{\tt \Large{
Prelude> length (1, 2)\\
}}
\end{frame}

\begin{frame}[fragile]
\frametitle{The Wat In Question}
{\tt \Large{
Prelude> length (1, 2)\\
1\\
Prelude> 
}}
\end{frame}

\begin{frame}
\frametitle{Wat Definition}
\begin{center}
{\Huge This is not a Wat!}

This behavior is consitent with the rest of the language.
\end{center}
\end{frame}

\begin{frame}[fragile]
\frametitle{Working Through the Logic}
{\Large
Length gives the number of elements held by a container.

\vspace{20pt}

{\tt
Prelude> length [ 1, 2, 3 ]\\
3\\
Prelude> 
}

\vspace{20pt}

Length just folds over the container, counting the elements as they go by.
}
\end{frame}

\begin{frame}
\frametitle{Working Through the Logic}
\begin{center}
{\Large 
Can you have a container that can \textsl{not} hold an arbitrary number
of elements, but can only hold at most one?
}
\end{center}
\end{frame}

\begin{frame}[fragile]
\frametitle{Working Through the Logic}
{\Large
{\tt Maybe} works as a "at most one" container:

\vspace{20pt}

{\tt
Prelude> length Nothing\\
0\\
Prelude> length (Just "foo")\\
1\\
Prelude> fmap show Nothing\\
Nothing\\
Prelude> fmap show (Just 1)\\
Just "1"\\
}}
\end{frame}


\begin{frame}
\frametitle{Working Through the Logic}
{\Large
You can partially apply multi-param types just like functions.

So {\tt Either String } works as a "at most one" container:

{\tt Prelude> foldr (+) 0 (Left "foo") \\
0 \\
Prelude> foldr (+) 0 (Right 2) \\
2 \\
Prelude> fmap show (Right 3) \\
Right "3" \\
Prelude> \\
}}
\end{frame}

\begin{frame}[fragile]
\frametitle{Working Through the Logic}
{\Large
It is foldable, there for length is defined on it:

\vspace{32pt}

{\tt
Prelude> :t length \\
length :: Foldable t => t a -> Int \\
Prelude> length (Left "error message") \\
0 \\
Prelude> length (Right 1) \\
1 \\
Prelude> 
}}
\end{frame}

\begin{frame}
\frametitle{Important Concept}
\begin{center}
\textbf{\Huge Tuples}

\vspace{20pt}

\textbf{\Huge Are}

\vspace{20pt}

\textbf{\Huge Not}

\vspace{20pt}

\textbf{\Huge Lists}

\end{center}
\end{frame}

\begin{frame}
\frametitle{Important Concept}
{\Large
\begin{itemize}
\item {\tt (,)} is a type, \textbf{just like Either}
\item {\tt (,)} takes two type parameters, \textbf{just like Either}
\item {\tt (,)} as a type can be partially applied, \textbf{just like Either}
\end{itemize}

\vspace{20pt}

So, if we apply {\tt (,)} to one type (like we do with Either), what do we get?
}
\end{frame}

\begin{frame}
\frametitle{Important Concept}
\begin{center}
{\Large
We get a container that holds exactly one value, as the second value of the tuple.
}
\end{center}
\end{frame}

\begin{frame}[fragile]
\frametitle{Tuples are a Container}
{\Large
{\tt
Prelude> foldr (+) 0 (1, 2) \\
2 \\
Prelude> foldr (+) 0 ("foo", 2) \\
2 \\
Prelude> fmap show (1, 2) \\
(1,"2") \\
Prelude> fmap show (False, 2) \\
(False,"2") \\
Prelude> length (False, 2) \\
1 \\
Prelude>  \\
}}
\end{frame}

\begin{frame}
\frametitle{Tuples are a Container}
\begin{center}
What value did you expect {\tt length} to return?

\vspace{32pt}

nil?

\vspace{32pt}

NaN?
\end{center}
\end{frame}
\begin{frame}
\begin{center}\tt \Huge{</RANT>}\end{center}
\end{frame}


\end{document}



